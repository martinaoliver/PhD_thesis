% Chapter Template

\chapter{Gene circuit design and parametrisation}

\section{Building model describing synthetic gene circuit}
In this section, I present the model that describes the synthetic gene circuit engineered in ~\cite{Tica2020}.
A PDE system is used to describe the reaction and diffusion terms, which describe change in concentrations of molecules in time and space.
This model includes constant background production, activator/repressor-regulated production using Hill terms, and the effect of tuning molecules aTc and IPTG on the circuit and diffusion.
The dynamics of the protein (X) and diffuser (U) species of the gene circuit are modeled as

\begin{subequations}\label{eq:Generalised protein and diffuser pde}
\begin{equation}
    \pdv{[X]}{t}= b_{X}+V_{X}\cdot\frac{1}{1+\left(\frac{K{A}}{[A]}\right)^{n_{A}}}\cdot\frac{1}{1+\left(\frac{[I]}{K_{I}}\right)^{n_{I}}}-\mu_{X}\cdot[X]
    \label{eq:Generalised protein}
\end{equation}

\begin{equation}
    \pdv{[U]}{t} = k_{1}\cdot[A] - \mu_{U}\cdot[U] + D_{U}(\partial_{xx} + \partial_{yy})[U]
    \label{eq:generalised diffuser}
\end{equation}
\end{subequations}

This results in a PDE model with eight equations that correspond to the two diffusers (pC and OC14) and six protein species of the circuit: RpaI, CinI, TetR, LacI, cI* and cI. This PDE model is reduced to a 6-equation system by assuming quasi-steady state for the diffusers, with production and degradation kinetics much faster than those of proteins. Finally, the model is non-dimensionalised to enable its parametrisation using liquid culture dose response data.
In the subsections below, the different terms of Eq ~\eqref{eq:Generalised protein and diffuser pde} are discussed. Furthermore, the process of model reduction, non-dimensionalisation and fitting is also described.


\subsection{Protein equations and gene regulation}
The rate of protein production can be defined as
\begin{equation}
    V = V_{max} \cdot \theta
\end{equation}
where $\theta$ represents the fractional activation of the system. Full activation of protein production is denoted as $\theta=1$ while full inhibition is represented by $\theta=0$.
$V_{\max}$ is the maximal rate of expression.
$\theta$ can be represented by a Hill function to describe cooperative binding, derived using the law of mass action with all-or-none binding to multiple binding sites ~\parencite{Weiss1997}.
We further apply the quasi-steady state assumptions for activator and inhibitor binding to the promoter, as well as for the mRNA dynamics, as these timescales are much faster than protein production ~\parencite{Andersen1998, Bremer2008}.
This leads to the following expression of $\theta$ for non-competitive activation (A) and inhibition (I)
\begin{equation}\label{eq:\theta}
    \theta  = \frac{1}{1+\left(\frac{K_{A}}{[A]}\right)^{n_{A}}} \cdot \frac{1}{1+\left(\frac{[I]}{K_{I}}\right)^{n_{I}}}
\end{equation}
This Hill function is used in the Eq ~\eqref{eq:Generalised protein} and describes promoter activity as a function of the two inputs [A] and [I], where $K_{A}$ and $K_{I}$ are half-activation/inhibition concentrations, $n_{A}$ and $n_{I}$ are the Hill coefficients.
Additionally, most promoters are leaky, which we account for by introducing a small rate of background production $b_{X}$. $b_{X}$ corresponds to the first term in Eq ~\eqref{eq:Generalised protein}
\subsection{Diffuser equations}
Diffusers are synthesised by enzymes, and then bound to receptors to activate gene expression by binding to the promoter. %check if its true they bind to the promoter.
The circuit receptors are expressed constructively from a low-copy pCC1 plasmid, and are therefore modelled with a constant concentration.
Quasi-steady state was assumed for the very fast equilibrium that forms between the receptor and the diffusers. %The receptor-inducer and receptor-promoter binding equilibria were modelled with mass action kinetics.
The enzymatic production of the diffusers was modelled with a simple linear production term dependent on synthesis enzyme concentration, and a rate constant.
It was assumed that the precursor substrate concentration is in excess and does not influence the reaction rate.
\subsection{Degradation}
All the species of the circuit were modelled to undergo linear (first order) degradation
\begin{equation}
    -\mu_{P_{1}}[P_{1}]
\end{equation}

as seen in ~\eqref{eq:Generalised protein and diffuser pde}.
The degradation rate parameters for the protein species are readily available in the literature
%(Table S2). They are scale-free parameters that only depend on time, which makes them more translatable between different experimental contexts and units of measurement.

\subsection{Tuning gene expression: aTc regulation of TetR and IPTG regulation of LacI }
The circuit was designed so it can be tuned in a variety of ways.
Experimentally, this tuning was used to achieve parameter combinations that are more favourable for patterning.
The tuning can be performed with: aTc, IPTG and DAPG.
aTc binds to TetR and inactivates it.
Only free, unbound TetR can bind TetO and inactivate the expression of cI*.
The binding of aTc to TetR is modelled by a reversible equilibrium.
The affinity of binding is given by the kon and $k_{off}$ rate constants.


\begin{equation}
    naTc + TetR_{free} \xrightleftharpoons[k_{off}]{k_{on}} TetR \mhyphen naTc
\end{equation}

This equilibrium happens at much faster rates than the protein production and degradation reactions of the model.
For this reason, quasi-steady state is assumed

\begin{equation}
    \pdv{TetR \mhyphen naTc}{t} = k_{on}[TetR_{free}][aTc]^n - k_{off}[TetR \mhyphen naTc] \approx 0
\end{equation}

By rearranging terms and using $K_{d} = k_{off}/k_{on}$, a simplified expression for $[TetR_{free}]$ is obtained, which is then used in the model

\begin{equation}
[TetR_{free}] = K_{D}\cdot \frac{[TetR \mhyphen naTc]}{[aTc]^n}
\end{equation}

However, because $[TetR-n_{aTc}]$ is unknown, the $[TetR_{total}]$ and $[TetR_{free}]$ are used instead
\begin{equation}
[TetR \mhyphen naTc] = [TetR_{total}] - [TetR_{free}]
\end{equation}
to obtain the following expression

\begin{subequations}
    \begin{equation}
    [TetR_{free}] = K_{D} \cdot \frac{[TetR_{total}] - [TetR_{free}]}{[aTc]^n} \Longrightarrow
    \end{equation}
    \begin{equation}
    [TetR_{free}] (1+\frac{K_{D}}{[aTc]^n}) = \frac{K_{D}\cdot [TetR_{total}]}{[aTc]^n}\Longrightarrow
    \end{equation}
    \begin{equation}
    [TetR_{free}] = \frac{K_{D}\cdot[TetR_{total}]}{[aTc]^n+K_{D}} = \frac{[TetR_{total}]}{1+\frac{[aTc]^n}{K_{D}}}
    \end{equation}
\end{subequations}

which says that the concentration of unbound TetR is given by ~$[aTc]$, ~$[TetR_{total}]$ and the equilibrium constant ~$K_{D}$.
For non-cooperative systems where ~$n=1$, ~$K_{D}=[L_{half}]$.
However, for cooperative systems where ~$n>1$, ~$[L_{half}] = \sqrt[n]{K_{D}}$.
We define a new variable ~$K_{A} = \sqrt[n]{K_{D}} \leftrightarrow K_{D} = K^n_{A}$ that we use to replace ~$K_{D}$ in the following expression, where ~$K^n_{A} = K_{TetR \mhyphen naTc}{[aTc]^n}$.
This was also done for other ~$K_{D}$ parameters of the model.
\begin{equation}
[TetR_{free}] =  \frac{[TetR_{total}]}{1+(\frac{[aTc]}{K_{TetR \mhyphen aTc}})^{n_{aTc}}}
\end{equation}
Parameter ~$K_{TetR \mhyphen aTc}$ is the binding affinity of aTc to TetR, whereas $n_{aTc}$ is the cooperativity of TetR and aTc binding.

\para
This expression for unbound TetR ($[TetR_{free}]$ ), can then be introduced into the production rate of cI*. cI* has an operator sequence downstream of the promoter, TetO. This TetO is bound by TetR to inhibit expression of the mRNA. We will model the production of cI* dependent of TetR binding, using an inverse-Hill term that describes repression by this binding.

\begin{equation}
    \pdv{[cI*]}{t} = V_{max}\cdot \left(1+\left(\frac{[TetR_{free}]}{K_{TetR \mhyphen TetO}}\right)^{n_{TetO}}\right)^{-1} = V_{max} \cdot \left(1+\left(\frac{TetR_{total}}{K_{TetR \mhyphen TetO}\cdot(1+(\frac{aTc}{K_{aTc}})^{n_{aTc}})}\right)^{n_{TetO}}\right)^{-1}
\end{equation}
For simplification, we can simplify the denominator so

\begin{equation}
    \pdv{[cI*]}{t} = V_{max}\cdot \left(1+\left(\frac{[TetR_{free}]}{K_{TetR \mhyphen TetO \mhyphen aTc }}\right)^{n_{TetO}}\right)^{-1}
\end{equation}
where,
\begin{equation}
    K_{TetR \mhyphen TetO \mhyphen aTc }=K_{TetR \mhyphen TetO}\cdot(1+(\frac{aTc}{K_{aTc}})^{n_{aTc}})
\end{equation}

\TODO{Explain different params here above}

The parameter KTetR-TetO is the binding affinity of TetR to tetO, KtetR-aTc is the affinity of aTc to TetR, naTc is the Hill coefficient of aTc and TetR, whereas nda is the Hill coefficient of TetR to tetO.

The same logic can be applied for the IPTG regulating LacI inhibition, where unbound LacI can be or $[LacI_{free}]$ can be expressed as
\begin{equation}
[LacI{free}] =  \frac{[LacI_{total}]}{1+(\frac{[IPTG]}{K_{LacI \mhyphen IPTG}})^{n_{IPTG}}}
\end{equation}

LacI free will be then able to bind to LacO and inhibit RpaI and tetR production in the following manner

\begin{equation}
    \pdv{[RpaI*]}{t} = V_{max}\cdot \left(1+\left(\frac{[LacI_{free}]}{K_{LacI \mhyphen LacO}}\right)^{n_{LacO}}\right)^{-1} = V_{max} \cdot \left(1+\left(\frac{LacI_{total}}{K_{LacI \mhyphen LacO}\cdot(1+(\frac{IPTG}{K_{IPTG}})^{n_{IPTG}})}\right)^{n_{LacO}}\right)^{-1}
\end{equation}

The denominator is simplified as

\begin{equation}
    \pdv{[RpaI*]}{t} = V_{max}\cdot \left(1+\left(\frac{[LacI_{free}]}{K_{LacI \mhyphen LacO \mhyphen IPTG}}\right)^{n_{LacO}}\right)^{-1}
\end{equation}

where,

\begin{equation}
    K_{LacI \mhyphen LacO \mhyphen IPTG}={K_{LacI \mhyphen LacO}\cdot(1+(\frac{IPTG}{K_{IPTG}})^{n_{IPTG}})}\right)
\end{equation}


This tuning equations link back to the circuit equations, as $[TetR_{free}]$ and $[LacI_{free}]$ are different examples of $[I]$ in the gene production equation (Eq ~\eqref{eq:\theta}). Therefore, if $[I]$ is replaced by either, this can be represented in the $K_{I}$ term in this equation resulting in



%todo figure of inactivation and equations


\section{Build equations for synthetic circuit}
\section{Explore parameter space and optimise robustness}
\section{Parametrise model to constrain}

\section{Produce rings}
\section{Modelling bacterial colonies}
\subsection{openboundary circle shape growth noise}
\section{Explore patterning potential}
\subsection{general params}
\subsection{parametrised params}
