\chapter{Discussion and Conclusion}

%introduction
Engineering biological Turing patterns, a mechanism that can explain many patterns in biology, has long been one of the holy grails in synthetic biology.
This thesis is dedicated to developing a modelling approach that will assist in the experimental design of robust patterning experiments using Turing synthetic gene networks.
Initially, it adopts a theoretical perspective, investigating analytical and numerical methods for studying such patterns.
Furthermore, it aims to theoretically understand how biological factors, such as multistability, boundaries, growth, and domain shape, influence patterning.
Once this foundational knowledge is acquired, a model for the synthetic experiment is constructed, enabling the identification of robust parameter spaces conducive to Turing patterning.
Additionally, by employing numerical solvers tailored to our experimental setup, we seek to determine whether the observed patterns are exclusively Turing patterns or a result of a combination of various mechanisms.

In the first results chapter, our findings using numerical methods revealed instances where the assumption that only Turing instabilities can produce periodic patterns is not accurate.
For instance, we observed that multistability can lead to pattern formation in non-Turing steady states and can also result in transient patterns that disrupt Turing steady states.
Additionally, periodic patterns were noted in both Turing-Hopf and unstable steady states.
These discoveries suggest that high-throughput methods testing Turing pattern robustness may need to revise their underlying assumptions in future research.
Furthermore, these numerical methods allowed for an in-depth exploration of realistic biological effects, such as the influence of boundaries and growth on pattern formation and disruption.
Future research should focus on developing more refined classification methods and investigating various types of growth, similar to those observed in our colonies.
Lastly, this chapter also highlights how pattern wavelength, temporal scales, and shape can be tuned using insights from dispersion relation analysis and optimization methods.
Future research could focus on applying such optimizations to our experimental system.
This would enhance our understanding of how to adjust parameters to achieve varying wavelengths and time scales, as well as to transition from spot patterns to labyrinths.

In the second chapter, a reaction-diffusion model based on a synthetic gene circuit is introduced.
This synthetic RD circuit, engineered in~\cite{Tica2020}, is based on the architecture of the Turing topology \#1754 from~\cite{Scholes2019}.
The six-equation PDE model accounts for inducer-dependent gene expression, degradation, and diffusion.
Utilizing insights from the first chapter, an extensive high-throughput sampling of the parameter space for this model is conducted.
It is observed that the gene circuit can form Turing patterns, including Turing I and Turing I-Hopf instabilities.
Additionally, regions within the parameter space with an increased likelihood of patterning are identified.
These regions can be experimentally attained through several methods, including matching dose-response curves and adding aTc or DAPG.
However, the robustness of the system in these optimised regions is still relatively low.
Extended work is required to enhance further the likelihood of Turing pattern formation and achieve robust patterning experiments.
This may involve more than parameter tuning, as other effects such as growth, boundaries or noise could be key in explaining the robustness of patterns observed in nature.

Additionally, the large parameter space of the model is constrained by parameterising the system using liquid-culture data.
A new method for model parameterisation is introduced, involving the non-dimensionalisation of the PDE system to allow comparisons of protein concentrations with relative fluorescent units.
This method goes beyond pattern formation studies, as it could be useful to parametrise other ODE and PDE models in synthetic biology involving gene circuits.
Once the parameters are constrained, we look for Turing instabilities in the fits and find some.
Once the parameters are constrained, Turing instabilities in the fitted distributions are identified.
These Turing instabilities are theoretically the closest Turing parameter sets in parameter space to the experimental system.
Future work should focus on producing more accurate fits using RT-qPCR data, where dose-response curves for every molecular species in the circuit are developed.
Additionally, alternative methods such as Approximate Bayesian Computation could be explored, which allows for the incorporation of prior knowledge and the exploration of non-linear relationships between parameters.

The engineered genetic circuit was examined in both the second and third chapters.
The second chapter delved into high-throughput searches in the parameter space using linear stability analysis, while the third chapter compares patterns obtained \textit{in-vitro} and \textit{in-silico} through numerical methods.
Building on insights about optimising robustness from the second chapter, in this thesis periodic rings were produced in growing colonies under high aTc conditions with matched dose-response curves.
A specialized PDE solver was developed to simulate the shape and growth of these biofilms and the dynamics of the RD gene circuit.
Image recognition was applied to microscopy images to model static domains, and a cellular automaton was used for colony growth.
Future work could merge these techniques, using domains from image recognition as initial conditions for the cellular-automaton algorithm.

Further experiments were carried out by collaborators, yielding a diverse array of patterns in the colonies.
All patterns could be replicated with the masked PDE solver, including rings, spots and even labyrinths which were not found experimentally.
Stationary regular patterns could be modelled with Turing I instability and Turing I-Hopf instability models, in particular outer ring addition dynamics.
Non-stationary patterns could be modelled with Hopf instability models, which generated mostly interior ring growth dynamics.
Linear stability analysis searches revealed high robustness for Hopf instabilities but extremely low for Turing instabilities, making experimental observation of the latter theoretically impossible.
However, in the vicinity of a Turing parameter set, the robustness was much higher, indicating the possibility of repeatable pattern observation.
Future work to develop more sophisticated classification methods for numerical patterns could clarify if robustness in growing colonies exceeds predictions from linear stability analysis.
This would aid in understanding how various factors associated with colony growth contribute to the system's robustness in patterning.
Notably, certain colony-induced patterns, not found in the first chapter, were documented, suggesting that the growth type used in our experiments might facilitate pattern formation.
Further theoretical work, similar to that conducted in the first chapter, should be undertaken for this specific experimental system.


The outer ring addition dynamics generated by the Turing parameter set from the fitted distribution were studied in detail to further understand the circuit dynamics in colonies.
This parameter set was chosen to be explored more in detail as its Turing behaviour should be the closest to a Turing mechanism in our experimental circuit.
In particular, it was explored to determine if the underlying mechanism for outer ring addition in our colonies was indeed a Turing mechanism.
While the time-series data, irregular growth controls and plate size controls aligned with a Turing model, the inability to replicate rings in node deletion controls suggested a Turing model might not fully explain this phenomenon.
Periodic cold shocks appeared to induce outer ring addition patterns.
However, a single initial cold shock followed by constant temperature also resulted in periodic patterns.
This suggests the explanation is related to a hybrid of theories: a positional information mechanism where a prepattern is seeded by the cold shock, combined with a Turing pattern in which periodic rings self-assemble.
Spots, as non-symmetrical patterns independent of radial shape, challenged the necessity of a cold shock pre-pattern for periodic patterns to occur
The observation of labyrinths, which are also asymmetrical, would further disprove this hypothesis.
In any case, further research should investigate this hybrid prepattern-Turing mechanism by integrating the effects of cold shock into the reaction-diffusion dynamics.
The cold shock could be modelled in two ways: either by modifying the model parameters at a specific time point $t$ or by starting the simulation with the initial condition being the first stripe generated by the cold shock.
This pre-pattern could be generated using image recognition from microscopy images.
Implementing such a model could reveal whether pattern formation is more robust with the inclusion of this pre-pattern, thus indicating that robust morphogenesis is likely the result of a combination of multiple mechanisms.
Interestingly, although Alan Turing's work was based on a homogeneous starting condition, he recognized the biological unreality of this assumption as he stated: "Most of an organism, most of the time is developing from one pattern to another, rather than from homogeneity into a pattern" ~\parencite{Turing1952}.


This work has significantly contributed to the engineering of reaction-diffusion patterns in synthetic biofilms through the use of predictive models.
Engineering these biofilms is essential for deepening our understanding of developmental biology, especially regarding the mechanisms responsible for robust spatial organisation.
This research indicates that a plausible explanation could involve a combination of pre-patterns, Hopf oscillators, and Turing instabilities.
Looking ahead, there is a vast array of potential applications for synthetic patterns.
Future endeavours with this circuit will involve substituting fluorescent genes with those necessary for the synthesis of novel materials, thereby creating patterns through biomaterial deposition.
Other long-term applications could include the synthesis of tissues for regenerative medicine or the development of organoids for use in organ models or transplants.
