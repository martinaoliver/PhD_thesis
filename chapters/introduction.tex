% Chapter Template

\chapter{Introduction}

quick intro
\section{Background on biological patterning}
\subsection{Morphogenesis in nature, examples and function}
Periodic spatial structures in biology result from the emergent phenomena where a tissue of cells gains heterogeneity and complexity in the spatial domain following certain regularity/frequency.
These spatial structures can be widely observed in nature both in 2D such as the zebra's stripped skin or in 3D in the labyrinthian brain cortex.
Many other examples can be observed (See Figure x).
Understanding the mechanism behind biological pattern formation is still a central question which remains elusive in the fields of systems, developmental and synthetic biology.

This evolution from simplicity to intricacy isn't merely aesthetic; it often plays a pivotal role in the functioning of multicellular beings.
Delving deeper, these patterns can have strategic advantages.
Fractal-shaped bacterial colonies, for instance, maximize nutrient absorption ~\parencite{Matsushita1990}.
Furthermore, distinct colorations, such as the zebra's stripes or the mesmerizing eye-spots on butterfly wings ~\parencite{Blest}, serve to disorient predators—either by disrupting the prey's silhouette or suggesting the prey is part of a larger entity ~\parencite{Stevens2006}.
The spirals observed in phylotaxis offer plants a way to optimize sunlight capture, enhancing photosynthesis ~\parencite{Strauss2020}.

The evolutionary advantage conveyed by this level of spatial organisation  hints towards genetic networks being potentially responsible for the patterning mechanism.
A randomly found solution of evolution where a pattern occurs, could lead to the genetic networks driving these spatial arrangements to be evolutionary beneficial and therefore selected.
Comprehending these networks and the dynamics that lead to these spatial designs is essential.
This area of research not only tries to understand the dynamics of genetic networks, but seeks to uncover how on a molecular scale these mechanisms are reliable, accurate and robust to the noisy and exposed biological environment.
Studying this goes beyond understanding of developmental biology and morphogenesis.
It paves the way for pioneering work in biotechnological sectors, enabling the creation of intricately patterned tissues, efficient biofilms, or even organoids that hold potential in groundbreaking applications, such as tissue regeneration or organ implants ~\parencite{Scholes2017,Tan2018}.



\subsection{Patterning theories}
Numerous mechanisms have been proposed in the literature to explain the formation of biological spatial patterns.
Some of the most relevant ones can be categorized into mechanical instabilities and diffusion-based mechanisms~\parencite{kosmrlj2020_2}.
The mechanical instabilities include phenomena such as differential adhesion, branching or wrinkling where physical forces are described in the model~\parencite{Scholes2017}.
% TODO SEARCH OTHER INSTABILITIES

On the other hand, gradient or diffusion based mechanisms rely on cells in the tissue responding to a molecule which is spatially heterogeneous and therefore exhibiting different phenotypes in different regions of the tissue (e.g.\ black vs white in zebra).


\section{Reaction-diffusion}
\subsection{Types of reaction diffusion: Wolpert vs Turing}
Among these gradient-based mechanisms, notable examples include positional information, the clock-and-wavefront model, and reaction-diffusion theory ~\parencite{Wolpert1969, Baker2006, Turing1952}.
Both the positional information and clock-and-wavefront concepts are underpinned by an initial gradient of morphogens, interpreted by a genetic network to induce tissue patterns.
Sometimes, the origins of this gradient can be attributed to external factors such as ambient temperature, maternal impacts, or light exposure ~\parencite{Schier2009}.
However, in specific instances, the presence of an initial pre-pattern might be implausible, raising questions about how patterns spontaneously emerge from a prior uniform tissue.
Addressing this conundrum, the reaction-diffusion model offers an alternative as it doesn't need a pre-existing gradient and is self-organising ~\parencite{Kondo2010a}.
Given its attributes, the reaction-diffusion model might provide a more complete answer to the question of patterning in biology.
Nevertheless, it's pivotal to recognize that these mechanisms aren't strictly independent of each other.
The intricate nature of biological patterning might be best elucidated by an amalgamation of these theories, suggesting a blended framework may hold the answers ~\parencite{Green2015}. %todo discuss turings comment Turing himself recognized the biological unreality of this in stating that 'most of an organism, most of the time is developing from one pattern to another, rather than from homogeneity into a pattern' 1-34]
These diffusion based mechanisms will be the focus of this work as they provide a more holistic perspective on the intricacies of morphogenesis.
% TODO comment more on the fight between wolpert and turing and on the french flag model in general
\subsection{Turing patterns}
Reaction-diffusion systems were originally introduced in 1952 by Alan M. Turing in his paper “The chemical basis of morphogenesis” ~\parencite{Turing1952}.
In this article, he argues that pattern formation can be obtained when two morphogens interact with each other and diffuse at different
velocities across a tissue.
These morphogens are small diffusible molecules that could be hormones, skin pigments or gene regulators.
The resulting patterns, called Turing patterns, can have different shapes such as stripes, labyrinths, spots; which can be widely observed in natural systems.
The concept of Turing patterns was first introduced from a mathematical perspective and was modelled using space and time dependent partial differential equations (PDE). In the case of Turing’s seminal paper, the set of PDEs describes a two node network (Figure 1b) which has activation and inhibition terms, degradation terms and diffusion terms (Figure 1a).
The key feature of this system is that it can exhibit diffusion driven instabilities: Initially, the tissue has an homogeneous concentration of morphogen under which no diffusion occurs, and under this circumstances the system is stable and converges into an equilibrium state.
When biological noise is introduced, the heterogeneity of the system leads to morphogen diffusion.
Consequently, the diffusion leads to changes in the reaction rates and the system is pushed out of steady state leading to an unstable system.
This unstable system converges into a stable stationary periodic pattern which is a Turing pattern (Figure 1a-c).
This whole phenomena called the diffusion driven instability is the essence of Alan Turing’s paper and one of the most used mechanisms to explain morphogenesis.


\subsection{Gierer Meindhardt}
Altough Turing’s work was of key importance in the field of developmental biology, it was not well accepted by biologists due to different reasons.
Firstly, the mathematical complexity behind it made it inaccessible for many scientists in the field.
Furthermore, the network proposed, although it can generate patterns, is too simple to describe the complexity in biological systems and takes some unrealistic assumptions such as the presence of negative concentrations ~\parencite{Kondo2010a}.
To solve these two issues, Gierer and Meindhardt proposed a generalised version of Turing’s diffusion-driven-instabilities ~\parencite{Gierer1972}.
This alternative theory proposes that patterns can be obtained as long as short-range activation and long-range inhibition (SALI) is present, meaning we can go away from Turing’s equations and extend the theory to networks with different number of nodes, different network interactions and even different signal transduction ~\parencite{Murray1983, Rauch2004, Swindale1980}.
This step was necessary to be able to attribute patterning phenomena to more complex cellular and molecular interactions described by non-linear terms and larger networks with more nodes involved.
Examples of the first systems developed with non-linear terms are the typical Gierer-Meindhard model ~\parencite{Gierer1972}, Schnakenberg model ~\parencite{Schnakenberg1979}, as well as the Thomas model %todo cite (Thomas and Kernevez 1976). %todo fitz nughamo?
This work, shed some light into the logic behind patterning and allows to understand it intuitively.
As an initially homogeneous system is perturbed by biological noise, local peaks of morphogen concentration form.
These transient peaks lead to a local activation effect where the concentration of all reactants increases.
Due to the difference in diffusivity, the inhibitor reaches further away and long range inhibition is achieved.
Ultimately, this system settles into a stationary pattern with activation peaks and inhibition troughs ~\parencite{Gierer1972}(
This mechanism is explained in (Figure 1d).

%todo insert figure 1d
\subsection{Turing patterns in nature}
Various links have been made between biological patterns found in nature and Turing patterning networks.
As a basic example, seashell pigmentation and various types of fish skin have been replicated using simulations of Turing pattern systems.
Additionally, perturbation experiments in zebrafish’s skin are consistent with simulations of RD equations where the pattern regenerates in the exact same way after being physically disrupted both in-vivo and in-silico ~\parencite{Kondo2010a}.
Finally, molecules involved in patterning and morphogenesis have been shown to be part of networks with SALI characteristics.
Examples of these are Nodal \& Lefty in right left assymetry ~\parencite{Nakamura2006}, Bmp \& Sox9 \& Wnt in limb digit development ~\parencite{J.Raspopovic1*L.Marcon1*L.Russo1J.Sharpe12014} and finally Wnt \& Dkk involve in lung branching %todo cite (De Langhe et al 2005).
All these examples of the relationship between biological patterns and RD systems, SALI networks and diffusion-driven instabilities strongly suggest that the Turing mechanism is linked to the self-assembling and self-regenerative patterns observed in nature.
However, the arguments mentioned above make Turing’s mechanism purely a strong hypothesis of patterning and do not prove causation.
%todo , digit patterning in the developing limb bud is set off by signals coming from the apical epidermal ridge (Raspopovic et al., 2014).
%todo tooth patterning isthought to involve sequential activation of new primary enamelknots, through a mechanism that may involve both localmorphogen dynamics and inputs from adjacent tissues (Kavanagh et al., 2007).
%todo add fingerprints
%todo add Recent experimental findings (Raspopovic et al., 2014; Junget al., 1998; Sick et al., 2006; Economou et al., 2012; Nakamasu et al., 2009) have resulted in Turing patterns being widely accepted as an important mechanism for spatial patterning in developmen-tal processes.
%todo Feather patterning appears to similarly combine these mechanisms (Kavanagh et al., 2007; Ho et al.,2019).
%todo link to mathematical analysis These findings have raised questions about the keyfeatures – or design principles – underlying the Turing mechanismand its robustness (Maini et al., June 2012). A variety of theoreticalinvestigations aiming to answer these questions have beenreported since. However, most of these have focused on singlemodels (Bard and Lauder, 1974; Collier et al., 1996; Oster, 1988;Crampin et al., 2002). More recently, some large-scale studies havesystematically analysed large parts of possible design spaces,thereby providing a novel understanding on how common androbust Turing pattern mechanism are (Marcon et al., 2016; Zhenget al., 2016; Scholes et al., 2019). The majority of these theoreticalstudies have used differential equation models with continuousconcentrations.
\section{Mathematical analysis of Turing patterns}

\subsection{Analysing Turing patterns}
Turing patterns are commonly studied using mathematical tools to understand their features and behaviours.
The equations describing these reaction-diffusion systems are usually too complex to solve analytically as they contain non-linearities and partial differential terms.
For this reason, different methods must be used such as linear stability analysis (LSA) or numerical methods.
LSA provides information on how the stability of the system evolves as we go from a system without diffusion to a system with diffusion ~\parencite{Glendinning1994}.
For RD systems to produce Turing patterns, they must be stable without diffusion and become unstable as diffusion is turned on (Murray, JD 2002).
This stability profile gives rise to the name “diffusion-driven instabilities” which is an alternative name to Turing patterns (Schneider 2012).
Overall, this method can tell us whether a system is a pattern generator, the wavelength and the relative speed of pattern formation.
Alternatively, numerical methods can be used calculate the RD system at every time point and space point (Ramos 1983).
A visual solution is obtained which provides information on the shape, wavelength, evolution and amplitude of the pattern.
While numerical methods provide more information than LSA, they are more computationally expensive and can lead to numerical errors (Murray, JD 2002).
Both have their advantages or disadvantages and must be used in combination for an optimal study of RD systems and patterning.
%todo mention visual pde as a facilitator
\subsection{Fine-tuning problem}
Even though they have been successfully engineered in chemical systems, the complexity of biology means that Turing patterns have never been engineered in a genetic context. 
The difficulty of engineering Turing patterns can be attributed mainly to their high sensitivity to changes in parameters and the small fraction of the parameter space leading to Turing patterns. 
This was seen in studies such as Scholes et al 2019 where all 2-node and 3-node networks were studied using LSA to check for Turing patterning.
It was found that although 61\% of networks can produce Turing patterns, only 0.1\% of the parameter space is within Turing space. 
Similar results were obtained in Zheng et al. 2016 and Marcon et al. 2016. 
The issue of fine-tuning is further amplified by the lack of orthogonal small-diffusable regulators to build a tuneable synthetic pattering circuit. 
However, this issue is partially solved in (Oliver Huidobro et al 2021) where a deep literature search is carried out to find potential orthogonal candidates to build tuneable multi-diffuser RD systems.

\subsection{Robustness of Turing patterns in realistic systems: bigger networks, growth, boundaries, noise, agar thickness}
\begin{itemize}
    \item large networks
    \begin{itemize}
      \item \parencite{Zheng2016, Scholes2019} Larger networks are more robust to paramtere variations and their robustness is less dependent to specific diffusion rates (e.g. differential diffusivity). These are more realistic biological networks.
      \item  \parencite{Smith2018a} Studying larger networks [1] It is very compuationally expensive to explore these larger networks. some studies attempt this by simplifying big networks into smaller ones such as ~\parencite{Smith2018a} %dalchau we make no claims that the reduced model is an accurate description of the original system. we can speed up search of turing instabilitis as if we can find a pattern-forming parameter set for the reduced system,then it is simply a matter of checking the stability of the Jaco- bian of the full system to determine whether it, too, forms patterns with those parameters. Furthermore, if we can find a region of parameter space for which the reduced system is stable, then we know for certain that the® full system cannot form patterns in that region.
        \item [2], Studying large networks using random matrices. 6 diffusers show that relaxation in differential diffusion. Could be done with hundreds of nodes if linearised random matrix (non diffusing nodes). work in ~\parencite{Haas2021}
        \item Relaxation of differential beneficial is needed as many available diffusors available are predicted to diffuse at roughly equal speeds ~\parencite{Kondo2010a, huidobro}
    \end{itemize}


    \item cooperativity ~\parencite{Diambra2015a} Greater steepness in dose response curve (higher n) increases parameter space region and decreases differential diffusivity constraints.
    \item robust topologies and motifs: most robust topologies found in ~\parencite{Scholes2019}. This topology is in line with design principles found in other robustnes papaers try using as many AIJT topologies in the network as possible (coherent activation and inhibition) 2node core activation inhibition topology. ~\parencite{Zheng2016}. adding immobile node ~\parencite{Zheng2016,Diego2018} reduces need for differential diffusion.  %todo cite scholes most robust topology, msc thesis. %todo cite  Korvasová K, Gaffney EA, Maini PK, Ferreira MA, Klika V. 2015 Investigating the Turing conditions for diffusion-driven instability in the presence of a binding immobile substrate. J. Theoret. Biol. 367, 286–295. (doi:10.1016/j.jtbi.2014.11.024) --> relaxation of the constraint of different morphogen diffusivities
    \item
    \item discrete models  %todo talk about discrete models (discrete lattice gas cellular automaton (LGCA) framework.)concentration discreetenes: it confines the concentrations to a small number of discrete values (four in our case) and comprises discrete maps between these states rather than continuous kinetic parameters.Moreover, we found these five topologies to be more robust in the LGCA framework than they appear to be in the continuous case. ~\parencite{Leyshon2021} %todo cite elowitz? spatial discreteness: single diffuser patterns. patterns occur if space is discretized to cell size.
    \item growth. Fixed domains were commonly assumed in developmental biology, as pattern forming processes (e.g reaction-diffusion of chemical species) occur at a much faster time-scale than tissue growth %todo add ref
    However, growth has been shown to add robustness to the pattern morphology in various biological examples the skin pattern of angelfish \textit{Pomacanthus imperator}. As the size of the
    organism incresed, new stripes would appear between the old ones, robustly maintaining not only the pattern morphology but also a near constant wavelength. This behaviour can be replicated with a reaction-diffusion model in a growing domain. If the same system is modelled in a fixed domain, the pattern morphology becomes overly sensitive to the initial conditions and leads to different stripes and spot patterns that differ to the biological system ~\parencite{Kondo2010a}. It is important to note that this robust stripe doubling has its caveats as it mibht break down under specific growth rates ~\parencite{Maini2012}.
    The importance of growth on pattern morphology was also observed in ~\parencite{Konow2019}, where they use a chemical reaction-diffusion system that grows over time. Slow growth rates were likely to produce rings that were added on the inside of the disk as it grew (inner ring addition), while fast growth rates produced rings on the outside (outer ring addition). Intermediate growth rates formed labyrinthine structures. This was further verified by the model.
    In terms of parametric robustness, ~\cite{gaffney2010} studied how the region of the parameter space changes when slow isotropic growth is considered. Interesingly, introducing growth allows certain network topologies to form Turing patterns, which would not without growth. For example, activator-activator networks leading to Turing instabilities, which they call domain-growth induced Turing instability. Additionally, they find short-range activation and long-range inhibition which can also produce Turing patterning in growing domains. They also show how increase of growth rates can increase the Turing parameter space under exponential growth in Turing patterning networks. They also test logistic and linear growth. However, the analytical analysis used in ~\cite{gaffney2010} only holds when assuming slow isotropic growth and will break under faster growths. This analysis is revisited in ~\cite{Klika2017} where they manage to study faster growing domains (still isotropic). However, they observe that the conditions for Turing instabilities are more complex with higher growth rates and therefore more difficult to dstudy than the ones in ~\cite{gaffney2010}. This leads to high throughput studies of the parameter space and parametric robustness difficult. Finally, other types of growth such as anisotropic growth ~\parencite{Krause2019} are studied but again they do not study parametric robustness in detail.
    \item
    \item
    \item
    \item
   %todo summary morphological and parametric robustness.
    morphological robustness: It is already well-known that in Turing models, the
    pattern selection process is very sensitive to initial conditions, scale, geometry, and
    parameter variation.


    \item
    \begin{itemize}
        \item boundaries. Indeed, the idea that biological systems exchange material with their immediate environment seems likely in many situations. Therefore no-flux boundary conditions might be irrealistic.  (point sources in butterfly wings %todo cite murray 1981)

    also boundary between two different types of cells may give rise to morphogen production at the boundary interface %todo ~\parencite{meinhardt1981}.
    nonhomogeneous boundary conditions give rise to solutions which are less sensitive to changes in the domain size, different initial conditions and perturbations in model parameters ~\parencite{Arcuri1986}.
        \item Mixed boundary conditions are also studied in ~\parencite{Dillon1994} which can also reduce the sensitivity of patterns to initial conditions, domain changes and allow for a larger ratio of diffusion coefficients . mixed boundaries occur when different speciers are subject to different boundary conditions (e.g.A is subject to Neumann conditions, whereas B satisfies Dirichlet conditions). also mentioned in ~\parencite{Krause2021}.
        \item Mixed bounndary (dirilichet and Neumann) driven instability %todo cite myercough maini 1997 and potentially adapt image

        \item (3) Boundary Conditions Cause Different Generic Bifurcation Structures in Turing Systems: Neumann boundary conditions can increase the Turing parameter space, meaning they are parametrically more robust. \parencite{Woolley2022}. %todo potentially ignore this paper as it goes against the rest and i dont understand it very well. 

        \item (4) )isolating patterns in open reaction-diffusion patterns
        \item  (5) It has been shown that imposing different boundary conditions, such as homogeneous Dirichlet, can greatly enhance the robustness of patterns in a Turing system by selecting, preferentially, certain modes at the expense of other modes which are no longer admissible.%todo~\parencite{pattern formation in generalized turing systems}
    \item Noise. In certain non-Turing parameter regimes, noise can drive diffusion-driven instablities. If there is a stable mode with eigenvalues slightly below zero, noise can destabilize this mode leading to a noise-induced instability and therefore the emergence of a pattern This might be a source of parametric robustness, producing patterns where deterministic systems do not predict one. Additionally, stochastic Turing patterns do not require large differential diffusivity between activator and inhibitor. However, these patterns are not as ordered as deterministic Turing patterns.  %todo cite  (Butler and Goldenfeld, 2009;2011,  Biancalani et al., 2010)., %cite Karig et al 2018
    In terms of morphological robustness, deterministic Turing patterns were subjected to intrinsic noise using the same parameter region, and it was shown analytically that stochastically excited wave modes correspond exactly to their deterministic analogues, leading to a similar pattern. Additionally, since noise perturbs populations away from the steady state. patterns can form quicker in a stochastic system than its deterministic counterpart ~\parencite{Maini2012}
    \item delays. Having talked about realistic biological effects that contribute to pattern robustness, gene expression have been shown to do exactly the opposite. Transcription and translation are biochemical processes that con occur in the scales of minutes to hours (depending on the size of the genomic sequence). Therefore from a gene being activated to the protein being active there is a delay that is often not accounted when modelling patterning. The pattern morphology has been shown to be extremenly sensitive to delays including senstivity to the initial conditions in the final pattern, dramatic changes in the patterning lag under different delays and even pattern loss when delays are introduced.
    \item agar thickness: suggest that the agar layer should bemade as thin as possible to limit the impact on a system’s ability to pattern. There may also be opportunities to decrease the permeability into the bulk,η, by using thicker filter paper or modifying the pore size or density, which would also  reduce the negative impact of the bulk on pattern formation, though due to metabolic constraints (as the agar is primarily a nutrient) this too may be somewhat limited. ~\parencite{Krause2020}
\end{itemize}

\section{ Engineering reaction-diffusion patterns}
\subsection{previous attempts at engineering patterns}
Due to the tangled nature of biological systems, it is extremely hard to prove that Turing’s mechanism is behind biological patterning. 
For this reason, a bottom-up approach such as synthetic biology is required to prove that this mechanism can lead to patterning in biology. 
Using this approach, a system is built from first principles which is simpler, tuneable and insulated from the tangled genetic context. 
Furthermore, synthetic patterning systems provide a powerful framework for tissue engineering (Scholes and Isalan 2017).
Many synthetic pattern generators have been engineered which range from systems without diffusers that rely on growth effects (Potvin-Trottier et al 2016; Riglar et al 2019), to reaction diffusion systems relying on positional information (Barbier et al 2020; Basu et al 2005; Boehm et al 2018; Grant et al 2020; Kong et al 2017; Schaerli et al 2014). 
Some other engineered RD patterns do not rely on an initial gradient, however they are not able to form stationary periodic patterns (Cao et al 2016; Danino et al. 2010; Payne et al. 2013). 
More detail on the existing synthetic patterning systems can be found in
All of these systems, although extremely interesting, cannot be classified as Turing patterns as none can self-assemble into stationary periodic patterns. 
One of the closest attempts to synthetic Turing patterning is seen in ~\parencite{Karig2018} where stochastic Turing patterns are engineered using two orthogonal morphogens in E.coli. However, the patterns are not regular in wavelength, size and fluorescent intensity, which means they cannot explain the regular process of morphogenesis. Alternatively, Sekine et al. 2018 uses an already existing patterning network (Nodal and lefty) in an eukaryotic system to produce self-assembling periodic patterns. However, this type of pattern (called solitary localised structures) also varies in shape and size and is extremely sensitive to initial conditions. Altough elusive in synthetic biology, chemical Turing patterns have been successfully engineered in systems such as the chlorite-iodide-malonic acid (CIMA) reaction (Castets et al 1990; Lengyel et al 1993) or the thiourea-iodate-sulfite (TuIS) reaction (Horvath et al. 2009) where regular- repeat stationary Turing patterns can be observed. These are tuneable systems which can be easily controlled (e.g with temperature), described with simple laws of mass action, have pre- dictable parameters and are isolated from external components such as cell burden, cross-talk etc (Butzin and Mather 2018; Ceroni et al 2015; Du et al 2020; Mu ̈ller et al 2019; Nielsen et al 2016).
%todo Although reaction-diffusion mechanisms have a simple network design, they exhibit unique self-organizing capabilities making them appealing for synthetic engineering (Diambra et al., 2015). Sofar, the synthetic implementation of reaction-diffusion systems has been impeded by the small pattern-forming parameter space of simple two-node models, their requirement for differential diffusivity (Carvalho et al., 2014), and a general gap between abstract models and real sender-receiver reaction-diffusion circuits (Marcon and Sharpe, 2012; Barcena Menendez et al., 2015). marcon high throughput

%todo comment that patterns are linked to gene expression because Modern experimental interrogations of suspected mor-phogen-based pattern formation systems, such as Nodal and Lefty zebrafish mesendodermal induction,use in situ hybridization [24–26]. This explicitly highlights local concentrations of specific mRNA tran- scripts and thus provides an indication of the rates of transcription of target genes, and emphasizes the role of gene expression in morphogenesis. Chen, Y. & Schier, A. F. 2002 Lefty proteins are long-
%range inhibitors of squint-mediated nodal signaling. Curr.
%Biol. 12, 2124–2128. (doi:10.1016/S0960-9822(02)01362-3)
%25 Jing, X. H., Zhou, S. M., Wang, W. Q. & Chen, Y. 2006
%Mechanisms underlying long- and short-range nodal
%signaling in Zebrafish. Mech. Dev. 123, 388–394.
%(doi:10.1016/j.mod.2006.03.006)
\subsection{engineering turing patterns in bacterial biofilms using exogenous circuit }
%todo comment on bespoke turing patterns
The Scholes et al 2019 study found that “network 3954” was the most robust in terms of kinetic and diffusion parameters (Figure 2a). A biological version of this network was implemented in E.coli using synthetic gene components. If this network can produce self-assembling periodic stationary patterns, it would prove that Turing patterns can form in biological systems supported by genetic interactions. An implementation of this circuit was done using novel diffusers found
8
in Meyer et al. 2019 and Du et al. 2020. Due to the lack of specific biological parts, the implementation resulted in a circuit with 6 molecular species instead of the original 3 node network explored in Scholes et al 2019. Details of this circuit can be seen in (Figure 2b) (Tica 2020).
%TODO insert figure 2d

\section{project overview}

\TODO