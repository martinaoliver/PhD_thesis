% Chapter Template

\chapter{Introduction}
quick intro
\section{Morphogenesis in nature. examples and function}
Periodic spatial structures in biology result from the emergent phenomena where a tissue of cells gains heterogeneity and complexity in the spatial domain following certain regularity/frequency.
These spatial structures can be widely observed in nature both in 2D such as the zebra's stripped skin or in 3D in the labyrinthian brain cortex.
Many other examples can be observed (See Figure x).
Undertstanding the mechanism behind biological pattern formation is still a central question which remains elusive in the fields of systems, developmental and synthetic biology.\\

This evolution from simplicity to intricacy isn't merely aesthetic; it often plays a pivotal role in the functioning of multicellular beings.
Delving deeper, these patterns can have strategic advantages. Fractal-shaped bacterial colonies, for instance, maximize nutrient absorption (Matsushita and Fujikawa 1990).
Furthermore, distinct colorations, such as the zebra's stripes or the mesmerizing eye-spots on butterfly wings, serve to disorient predators—either by disrupting the prey's silhouette or suggesting the prey is part of a larger entity (Blest 1957; Stevens et al. 2006).
The spirals observed in phylotaxis offer plants a way to optimize sunlight capture, enhancing photosynthesis (Strauss et al. 2020).\\

The evolutionary advantage conveyed by this level of spatial organisation  hints towards genetic networks being potentially responsible for the patterning mechanism.
A randomly found solution of evolution where a pattern occurs, could lead to the genetic networks driving these spatial arrangements to be evolutionary beneficial and therefore selected.
Comprehending these networks and the dynamics that lead to these spatial designs is essential.
This area of research not only tries to understand the dynamics of genetic networks, but seeks to uncover how on a molecular scale these mechanisms are reliable, accurate and robust to the noisy and exposed biological environment.
Studying this goes beyond understanding of developmental biology and morphogenesis.
It paves the way for pioneering work in biotechnological sectors, enabling the creation of intricately patterned tissues, efficient biofilms, or even organoids that hold potential in groundbreaking applications, such as tissue regeneration or organ implants (Scholes and Isalan 2017; Tan et al. 2018).\\



\section{Patterning theories}
Numerous mechanisms have been proposed in the literature to explain the formation of biological patterns.
Broadly, these can be categorized into mechanical instabilities and diffusion-based mechanisms \autocite{kosmrlj2020}.
The mechanical instabilities include phenomena such as differential adhesion, branching or wrinkling where physical forces are described in the model \cite{Scholes2017}.
% TODO SEARCH OTHER INSTABILITIES
On the other hand, gradient or diffusion based mechanisms rely on cells in the tissue responding to a molecule which is spatially heterogeneous and therefore exhibiting different phenotypes in different regions of the tissue (e.g. black vs white in zebra).\\

Among these gradient-based mechanisms, notable examples include positional information, the clock-and-wavefront model, and reaction-diffusion theory (Baker et al. 2006; Turing 1952; Wolpert 1969).
Both the positional information and clock-and-wavefront concepts are underpinned by an initial gradient of morphogens, interpreted by a genetic network to induce tissue patterns. Sometimes, the origins of this gradient can be attributed to external factors such as ambient temperature, maternal impacts, or light exposure (Schier 2009).
However, in specific instances, the presence of an initial pre-pattern might be implausible, raising questions about how patterns spontaneously emerge from a prior uniform tissue.
Addressing this conundrum, the reaction-diffusion model offers an alternative as it doesn't need a pre-existing gradient and is self-organising (Kondo and Miura 2010). Given its attributes, the reaction-diffusion model might provide a more complete answer to the question of patterning in biology. Nevertheless, it's pivotal to recognize that these mechanisms aren't strictly independent of each other. The intricate nature of biological patterning might be best elucidated by an amalgamation of these theories, suggesting a blended framework may hold the answers (Green and Sharpe 2015).
These diffusion based mechanisms will be the focus of this work as they provide a more holistic perspective on the intricacies of morphogenesis.

\section{Robustness issues}

\section{Realistic study of turing patterns to solve robustness: bigger networks, growth, boundaries, noise}
\section{ synthetic approaches (review)}

\section{project overview}

